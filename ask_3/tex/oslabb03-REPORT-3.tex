\documentclass[a4paper,10pt]{article} \usepackage{anysize}
\marginsize{2cm}{2cm}{1cm}{1cm}
%\textwidth 6.0in \textheight = 664pt
\usepackage{xltxtra}
\usepackage{xunicode} \usepackage{graphicx}
\usepackage{color} \usepackage{xgreek} \usepackage{fancyvrb}
\usepackage{minted}
\usepackage{listings}
\usepackage{enumitem} \usepackage{framed} \usepackage{relsize}
\usepackage{float} \setmainfont[Mapping=TeX-text]{CMU Serif}
\begin{document}

\begin{titlepage}
\begin{center}
\begin{figure}[t] 
     \includegraphics[scale=0.7]{title/ntua_logo}
\end{figure}
\begin{LARGE}\textbf{ΕΘΝΙΚΟ ΜΕΤΣΟΒΙΟ ΠΟΛΥΤΕΧΝΕΙΟ\\}\end{LARGE}
\vspace{2cm}
\begin{Large}
ΣΧΟΛΗ ΗΜ\&ΜΥ\\
ΤΟΜΕΑΣ ΤΕΧΝΟΛΟΓΙΑΣ ΕΠΙΚΟΙΝΩΝΙΩΝ, ΗΛΕΚΤΡΟΝΙΚΗΣ ΚΑΙ ΣΥΣΤΗΜΑΤΩΝ ΠΛΗΡΟΦΟΡΙΚΗΣ\\
ΕΡΓΑΣΤΗΡΙΟ ΨΗΦΙΑΚΩΝ ΣΥΣΤΗΜΑΤΩΝ\\
1\textsuperscript{η} ΑΣΚΗΣΗ\\
Ακ. έτος 2010-2011\\
\end{Large}
\vspace{5cm}
\Large Τμήμα Γ, Ομάδα 2\textsuperscript{η}\\
\vspace{1cm}
\begin{tabular}{l r}
\Large{Γερακάρης Βασίλης}&
\large{Α.Μ.: 03108137}\\
\Large{Λύρας Γρηγόρης}&
\large{Α.Μ.: 03109687}\\
\end{tabular}\\
\vspace{5cm}

\vfill
\large\today\\
\end{center}
\end{titlepage}



\section*{} \setcounter{section}{1}
\subsection{Υλοποίηση σημαφόρων με σωληνώσεις του UNIX} 
Ο πηγαίος κώδικας της pipesem.c που γράψαμε είναι ο παρακάτω:
\inputminted[linenos,fontsize=\footnotesize,frame=leftline]{c}{files/pipesem.c}

\emph{Ερωτήσεις}
\begin{enumerate}
\item Αν η διεργασία A τερματίσει πρόωρα τα παιδιά της (B,C) θα συνεχίσουν να
εκτελούνται και υιοθετούνται από την INIT. Όταν αυτά τερματίσουν, η INIT θα
κάνει wait και θα φύγουν από τη μνήμη.
\item Φαίνεται μία ακόμη διεργασία. Αυτή είναι η διεργασία που ξεκίνησε τα
forks δημιουργώντας την A.
\item Σε διαφορετική περίπτωση ένας χρήστης θα μπορούσε να εκτελέσει
απεριόριστες διεργασίες. Κάτι τέτοιο δεν επιτρέπεται, μιας και η ανεξέλεγκτη
κατανάλωση πόρων συστήματος μπορεί να καταστήσει το σύστημά μας μη
λειτουργικό, κάτι που ο διαχειριστής πρέπει να αποφύγει.\\
\end{enumerate}


\subsection{Παράλληλος υπολογισμός του συνόλου Mandelbrot} 
Ο πηγαίος κώδικας παρατίθεται παρακάτω:
\inputminted[linenos,fontsize=\footnotesize,frame=leftline]{c}{files/mandel.c}

Η έξοδος που προέκυψε κατά την εκτέλεση ήταν η εξής, με τους αντίστοιχους χρόνους πρίν και μετά την παραλληλία:
% GREG EDW THELOUME FWTO + TA TIME OUTPUTS

\emph{Ερωτήσεις}
\begin{enumerate}
\item Οι διεργασίες γεννιούνται κατά επίπεδο. Παρόλα αυτά, δεν είναι
εξασφαλισμένη η σειρά που θα εκτελεστούν οι διεργασίες. Έχει να κάνει με τη
χρονοδρομολόγηση (δηλαδή το D του προηγούμενου ερωτήματος θα μπορούσε να
εκτελεστεί πριν από το C αν η χρονοδρομολόγηση το ευνοούσε.
Ωστόσο το κβάντο χρόνου είναι τέτοιο που η εμφάνιση των
μηνυμάτων δημιουργίας μοιάζει σχεδόν σειριακή.
\item 2
\item 3
\item 4
\end{enumerate}

\subsection{Ταυτόχρονη πρόσβαση σε μοιραζόμενους πόρους} 
Ο πηγαίος κώδικας παρατίθεται παρακάτω:
\inputminted[linenos,fontsize=\footnotesize,frame=leftline]{c}{files/ask2-signals.c}
\pagebreak

Η έξοδος που προέκυψε κατά την εκτέλεση ήταν η εξής:
\inputminted[linenos,fontsize=\footnotesize,frame=leftline]{bash}{files/ask2-signals.out}

\emph{Ερωτήσεις}
\begin{enumerate}
\item Είναι σαφώς καλύτερη και πιο αξιόπιστη, καθώς δε στηρίζεται στη
χρονοδρομολόγηση των διεργασιών.
\item Προτού στείλουμε σήματα στα παιδιά πρέπει να έχουμε βεβαιωθεί πως αυτά
είναι σε ready state ώστε να δεχτούν σήμα. Αν το κάνουμε νωρίτερα αυτό θα
αγνοηθεί. Συνεπώς περιμένουμε μέχρι όλα τα παιδιά να κάνουν raise(SIGSTOP)
ώστε να δεχτούν το SIGCONT.
\end{enumerate}

\subsection{Προαιρετικές ερωτήσεις (bonus)} 

\begin{enumerate}
\item Monitors
%tha to kanoume telika???

\item Rand-fork
\end{enumerate}

\subsection{Bonus track: \\Οπτικοποίηση Mandelbrot Set με export σε εικόνα \& \\επέκταση παραλληλοποίησης σε πολλαπλούς υπολογιστές (νησιά)}
%Do your magic greg
CODE CODE CODE

\end{document}
