\documentclass[a4paper,10pt]{article} \usepackage{anysize}
\marginsize{2cm}{2cm}{1cm}{1cm}
%\textwidth 6.0in \textheight = 664pt
\usepackage{xltxtra}
\usepackage{xunicode} \usepackage{graphicx}
\usepackage{color} \usepackage{xgreek} \usepackage{fancyvrb}
\usepackage{minted}
\usepackage{listings}
\usepackage{enumitem} \usepackage{framed} \usepackage{relsize}
\usepackage{float} \setmainfont[Mapping=TeX-text]{CMU Serif}
\begin{document}

\begin{titlepage}
\begin{center}
\begin{figure}[t] 
     \includegraphics[scale=0.7]{title/ntua_logo}
\end{figure}
\begin{LARGE}\textbf{ΕΘΝΙΚΟ ΜΕΤΣΟΒΙΟ ΠΟΛΥΤΕΧΝΕΙΟ\\}\end{LARGE}
\vspace{2cm}
\begin{Large}
ΣΧΟΛΗ ΗΜ\&ΜΥ\\
ΤΟΜΕΑΣ ΤΕΧΝΟΛΟΓΙΑΣ ΕΠΙΚΟΙΝΩΝΙΩΝ, ΗΛΕΚΤΡΟΝΙΚΗΣ ΚΑΙ ΣΥΣΤΗΜΑΤΩΝ ΠΛΗΡΟΦΟΡΙΚΗΣ\\
ΕΡΓΑΣΤΗΡΙΟ ΨΗΦΙΑΚΩΝ ΣΥΣΤΗΜΑΤΩΝ\\
1\textsuperscript{η} ΑΣΚΗΣΗ\\
Ακ. έτος 2010-2011\\
\end{Large}
\vspace{5cm}
\Large Τμήμα Γ, Ομάδα 2\textsuperscript{η}\\
\vspace{1cm}
\begin{tabular}{l r}
\Large{Γερακάρης Βασίλης}&
\large{Α.Μ.: 03108137}\\
\Large{Λύρας Γρηγόρης}&
\large{Α.Μ.: 03109687}\\
\end{tabular}\\
\vspace{5cm}

\vfill
\large\today\\
\end{center}
\end{titlepage}



\section*{} \setcounter{section}{1}
\subsection{Υλοποίηση σημαφόρων με σωληνώσεις του UNIX} 
Ο πηγαίος κώδικας που γράψαμε είναι ο παρακάτω:
\inputminted[linenos,fontsize=\footnotesize,frame=leftline]{c}{files/pipesem.h}
\inputminted[linenos,fontsize=\footnotesize,frame=leftline]{c}{files/pipesem.c}

\subsection{Παράλληλος υπολογισμός του συνόλου Mandelbrot} 
Ο πηγαίος κώδικας που γράψαμε είναι ο παρακάτω:
\inputminted[linenos,fontsize=\footnotesize,frame=leftline]{c}{files/mandel.c}

\subsection{Ταυτόχρονη πρόσβαση σε μοιραζόμενους πόρους}
Ο πηγαίος κώδικας που γράψαμε είναι ο παρακάτω:
\inputminted[linenos,fontsize=\footnotesize,frame=leftline]{c}{files/procs-shm.c}

\end{document}
