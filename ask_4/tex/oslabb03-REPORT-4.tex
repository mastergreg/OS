\documentclass[a4paper,10pt]{article} \usepackage{anysize}
\marginsize{2cm}{2cm}{1cm}{1cm}
%\textwidth 6.0in \textheight = 664pt
\usepackage{xltxtra}
\usepackage{xunicode} \usepackage{graphicx}
\usepackage{color} \usepackage{xgreek} \usepackage{fancyvrb}
\usepackage{minted}
\usepackage{listings}
\usepackage{enumitem} \usepackage{framed} \usepackage{relsize}
\usepackage{float} \setmainfont[Mapping=TeX-text]{FreeSerif}
\begin{document}

\begin{titlepage}
\begin{center}
\begin{figure}[t] 
     \includegraphics[scale=0.7]{title/ntua_logo}
\end{figure}
\begin{LARGE}\textbf{ΕΘΝΙΚΟ ΜΕΤΣΟΒΙΟ ΠΟΛΥΤΕΧΝΕΙΟ\\}\end{LARGE}
\vspace{2cm}
\begin{Large}
ΣΧΟΛΗ ΗΜ\&ΜΥ\\
ΤΟΜΕΑΣ ΤΕΧΝΟΛΟΓΙΑΣ ΕΠΙΚΟΙΝΩΝΙΩΝ, ΗΛΕΚΤΡΟΝΙΚΗΣ ΚΑΙ ΣΥΣΤΗΜΑΤΩΝ ΠΛΗΡΟΦΟΡΙΚΗΣ\\
ΕΡΓΑΣΤΗΡΙΟ ΨΗΦΙΑΚΩΝ ΣΥΣΤΗΜΑΤΩΝ\\
1\textsuperscript{η} ΑΣΚΗΣΗ\\
Ακ. έτος 2010-2011\\
\end{Large}
\vspace{5cm}
\Large Τμήμα Γ, Ομάδα 2\textsuperscript{η}\\
\vspace{1cm}
\begin{tabular}{l r}
\Large{Γερακάρης Βασίλης}&
\large{Α.Μ.: 03108137}\\
\Large{Λύρας Γρηγόρης}&
\large{Α.Μ.: 03109687}\\
\end{tabular}\\
\vspace{5cm}

\vfill
\large\today\\
\end{center}
\end{titlepage}


\section*{Ερωτήσεις}
\setcounter{section}{3}
\subsection{Άσκηση 1.1}
\subsubsection{}
Στην περίπτωση που έρθει ένα σήμα κατά την εκτέλεση της ρουτίνας ενός signal
handler, αυτό αγνοείται προσωρινά, αφού γίνονται masked όταν ένας από τους 2
τρέχει.
\subsubsection{}
Τα SIGCHLD συνήθως προέρχονται από running παιδιά που δέχονται SIGSTOP από τον
scheduler όταν τελειώσει το κβάντο χρόνου τους (οπότε φέρνουμε για εκτέλεση
την επόμενη διεργασία) ή SIGCONT (όταν μια ready διεργασία επιλεγεί για
εκτέλεση, που το αγνοούμε). Επίσης έρχεται 1 σήμα SIGCHILD (κατα τη δημιουργία
μιας διεργασίας, όπου κάνει raise SIGSTOP στον εαυτό της) και 1 όταν κάνει
έξοδο. Με τη χρήση μιας global μεταβλητής current\_proc, όπου διατηρούμε το pid
της διεργασίας που εκτελείται κάθε στιγμή, μπορούμε να
\subsubsection{}
b

\subsection{Άσκηση 1.2}
\subsubsection{}
Εμφανίζεται πάντα ο φλοιός ως current\_proc, αφού προκειμένου να εκτελέσει τις
εντολές που του δώθηκαν (μαζί και την 'p'), πρέπει να είναι running. Δε
γίνεται να αποφευχθεί αυτό, αφού οι εντολές του πρέπει αυστηρά να εκτελούνται
μέσα στο κβαντο χρόνου που του αντιστοιχεί.
\subsubsection{}
Αν δε γινόταν απενεργοποιήση των σημάτων πριν την υλοποίηση των συναρτήσεων
του φλοιού θα ήταν πιθανό να έρθει ένα σήμα που θα μεταβάλει την κατάσταση της
ουράς εκτέλεσης την ώρα που ο φλοιός (για παράδειγμα) διέγραφε μια διεργασία.
Αυτό το race condition μπορούσε να προκαλέσει σφάλματα όπως διαγραφή λάθος
διεργασίας, ή (χειρότερα) κατάρρευση της ουράς με λάθος ανάθεση των pointers
στις επόμενες διεργασίες.

\subsection{Άσκηση 1.3}
\subsubsection{}
Θα ήταν δυνατόν να λιμοκτονήσει μια low\_priority διεργασία αν συνεχώς
δημιουργούσαμε και θέταμε ως high priority διεργασίες. Αφού η λίστα των υψηλής
προτεραιότητας διεργασιών θα είχε πάντα στοιχεία, ο χρονοδρομολογητής δε θα
επέλεγε ποτε τις διεργασίες με χαμηλή προτεραιότητα.

\section*{Προαιρετικές Ερωτήσεις}
\setcounter{section}{4}
\setcounter{subsection}{0}
\subsection{}
a
\subsection{}
Θα ήταν δυνατόν να διατηρείται στη λίστα με τις low\_priority διεργασίες ένας
counter που θα δείχνει πόσα κβάντα χρόνου έχουν περάσει από την δημιουργία
τους, ενώ υπάρχουν high\_priority διεργασίες. Ο counter αυτός θα αυξανόταν μέσα
στον SIGALRM handler (δίνοντας βέβαια αρκετή καθυστέρηση στο πρόγραμμα). Όταν
μία διεργασία έφτανε ένα προκαθορισμένο αριθμό στον counter, θα ανέβαινε σε
high\_priority ώστε να αρχίζει να εκτελείται.
\subsection{}
c


\vspace{1cm}
\def\thesubsection {Άσκηση \arabic{section}.\arabic{subsection}}
\section*{Πηγαίος κώδικας ασκήσεων}
\emph{Κοινή βιβλιοθήκη για διαχείρηση διπλά συνδεδεμένης, κυκλικής λίστας}
\inputminted[linenos,fontsize=\footnotesize,frame=leftline]{c}{files/queue.h}
\inputminted[linenos,fontsize=\footnotesize,frame=leftline]{c}{files/queue.c}
\setcounter{section}{1}
\setcounter{subsection}{0}
\subsection{}
\inputminted[linenos,fontsize=\footnotesize,frame=leftline]{c}{files/scheduler_1.1.c}
\subsection{}
\inputminted[linenos,fontsize=\footnotesize,frame=leftline]{c}{files/scheduler-shell_1.2.c}
\subsection{}
\inputminted[linenos,fontsize=\footnotesize,frame=leftline]{c}{files/scheduler-shell_1.3.c}

\section*{Παραδείγματα εκτέλεσης των προγραμμάτων}
\def\thesubsection {Πρόγραμμα \arabic{section}.\arabic{subsection}:}
\setcounter{subsection}{0}
\subsection{Scheduler}
a
\subsection{Scheduler with shell}
b
\subsection{Scheduler with shell \& priorities}
c

\end{document}
