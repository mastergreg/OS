\documentclass[a4paper,10pt]{article} \usepackage{anysize}
\marginsize{2cm}{2cm}{1cm}{1cm}
%\textwidth 6.0in \textheight = 664pt
\usepackage{xltxtra} 
\usepackage{xunicode} \usepackage{graphicx}
\usepackage{color} \usepackage{xgreek} \usepackage{fancyvrb}
\usepackage{minted}  
\usepackage{listings}
\usepackage{enumitem} \usepackage{framed} \usepackage{relsize}
\usepackage{float} \setmainfont[Mapping=TeX-text]{Times New Roman}
\begin{document}

\begin{titlepage}
\begin{center}
\begin{figure}[t] 
     \includegraphics[scale=0.7]{title/ntua_logo}
\end{figure}
\begin{LARGE}\textbf{ΕΘΝΙΚΟ ΜΕΤΣΟΒΙΟ ΠΟΛΥΤΕΧΝΕΙΟ\\}\end{LARGE}
\vspace{2cm}
\begin{Large}
ΣΧΟΛΗ ΗΜ\&ΜΥ\\
ΤΟΜΕΑΣ ΤΕΧΝΟΛΟΓΙΑΣ ΕΠΙΚΟΙΝΩΝΙΩΝ, ΗΛΕΚΤΡΟΝΙΚΗΣ ΚΑΙ ΣΥΣΤΗΜΑΤΩΝ ΠΛΗΡΟΦΟΡΙΚΗΣ\\
ΕΡΓΑΣΤΗΡΙΟ ΨΗΦΙΑΚΩΝ ΣΥΣΤΗΜΑΤΩΝ\\
1\textsuperscript{η} ΑΣΚΗΣΗ\\
Ακ. έτος 2010-2011\\
\end{Large}
\vspace{5cm}
\Large Τμήμα Γ, Ομάδα 2\textsuperscript{η}\\
\vspace{1cm}
\begin{tabular}{l r}
\Large{Γερακάρης Βασίλης}&
\large{Α.Μ.: 03108137}\\
\Large{Λύρας Γρηγόρης}&
\large{Α.Μ.: 03109687}\\
\end{tabular}\\
\vspace{5cm}

\vfill
\large\today\\
\end{center}
\end{titlepage}



\section*{} \setcounter{section}{1}
\subsection{Σύνδεση με αρχείο αντικειμένων} Ο πηγαίος κώδικας της main.c που
κληθήκαμε να γράψουμε ήταν ο εξής:

\inputminted[linenos]{c}{../zing/main.c}

Στη συνέχεια δημιουργήσαμε το makefile για  τη μεταγλώττιση του προγράμματος
με τα εξής περιεχόμενα:

\inputminted[linenos,obeytabs]{basemake}{../zing/makefile}

Τρέχοντας στο shell την εντολή make έχουμε την παρακάτω έξοδο
\begin{minted}[linenos]{text}
gcc -c main.c -o main.o -Wall -m32
gcc main.o zing.o -o main -Wall -m32
\end{minted}
και τη δημιουργία των αρχείων main.o και του εκτελέσιμου main.\\
Εκτελώντας το main, το πρόγραμμα δίνει την παρακάτω έξοδο:
\begin{minted}[linenos]{text}
oslabb03 ~/code/zing $ ./main
Hello oslabb03!
\end{minted}


\section*{Απαντήσεις στις θεωρητικές ερωτήσεις}
\begin{enumerate}
\item Η επικεφαλίδα που χρησιμοποιήσαμε περιέχει τις απαραίτητες  δηλώσεις για τη
διεπαφή των αρχείων κώδικα του προγράμματος μας. 
Η άσκηση αυτή μας παρείχε το object file zing.o , αλλά η συνάρτηση zing( )
δηλώνεται στο zing.h, χωρίς τη χρήση του οποίου δε θα μπορούσαμε να την
καλέσουμε επιτυχώς στη main.

\item Απαντήθηκε παραπάνω.

\item Αντί να έχουμε όλες τις συναρτήσεις σε ένα αρχείο θα μπορούσαμε να
χρησιμοποιούμε ένα αρχείο για κάθε συνάρτηση με το αντίστοιχο αρχείο
επικεφαλίδας. Έτσι η μεταγλώτισση θα γίνεται για κάθε αρχείο χωριστά. Συνεπώς
αλλάζοντας ένα αρχείο ο χρόνος μεταγλώττισης θα είναι μικρότερος. Επίσης με
αυτό τον τρόπο μπορούμε να κάνουμε παράλληλη μεταγώττιση αρχείων σε περίπτωση 
που το σύστημα μας δίνει αυτή τη δυνατότητα.

\item Στην περίπτωση αυτή βλέπουμε πως το αρχείο foo.c μεταγλωττίστηκε στο
αρχείο foo.c. Τώρα πλέον το foo.c είναι το εκτελέσιμο και ο πηγαίος κώδικας
χάθηκε.
\end{enumerate}


\pagebreak

\subsection{Συνένωση δύο αρχείων σε τρίτο}

Ο πηγαίος κώδικας που χρησιμοποιήσαμε αρχικά ήταν ο εξής:
\inputminted[linenos,fontsize=\footnotesize]{c}{../stage_1/fconc.h}

\inputminted[linenos,fontsize=\footnotesize]{c}{../stage_1/fconc.c}

\inputminted[linenos,fontsize=\footnotesize]{basemake}{../stage_1/makefile}


Η έξοδος της strace είναι η παρακάτω:
\lstinputlisting[numbers=left,numberstyle=\tiny,basicstyle=\footnotesize\ttfamily,breaklines=true]{../stage_1/strace_outfile}
\inputminted[linenos,obeytabs,fontsize=\footnotesize]{text}{../stage_1/strace_outfile}

\subsection{Bonus}
Ο πηγαίος κώδικας που χρησιμοποιήσαμε τελικά ήταν ο εξής:
\inputminted[linenos,fontsize=\footnotesize]{c}{../stage_2/fconc.h}
\inputminted[linenos,fontsize=\footnotesize]{c}{../stage_2/fconc.c}
\inputminted[linenos,fontsize=\footnotesize]{basemake}{../stage_2/makefile}
Η έξοδος της strace είναι η παρακάτω:
\lstinputlisting[numbers=left,numberstyle=\tiny,basicstyle=\footnotesize\ttfamily,breaklines=true]{../stage_2/strace_outfile}
Με τη χρήση του gdb στα αρχεία main.o και zing είχαμε την παρακάτω έξοδο.
\lstinputlisting[numbers=left,numberstyle=\tiny,basicstyle=\footnotesize\ttfamily,breaklines=true]{../zing/gdb1.txt}
\lstinputlisting[numbers=left,numberstyle=\tiny,basicstyle=\footnotesize\ttfamily,breaklines=true]{../zing/gdb2.txt}

\end{document}
